% \documentclass{article}
\documentclass[border=0mm, convert={density=300, quality=90, outext=.png}]{standalone}
% ===== Loading all packages =====
% === input, languages, layout ===
\usepackage[utf8]{inputenc}
\usepackage[T1]{fontenc}
\usepackage[ngerman]{babel}
\usepackage{textcomp}
\usepackage{url}
\usepackage{printlen}
\usepackage{lmodern}                % Latin Modern, bessere Fonts

% === mathematics, numbers, units ===
\usepackage{amsmath}
\usepackage{amssymb}
\usepackage{siunitx}
\usepackage{calc}                   % enables the calculation of distances etc.

% === figures, plots, drawing, ... ===
\usepackage[cmyk]{xcolor}
\usepackage{graphicx}
\usepackage{tikz}
\usepackage{pgfplots}
\usepackage{pgfplotstable}

% === other stuff ===
\usepackage{multirow}               % Table entries reaching over more than one line
\usepackage[
    layout = inline,
    final]{fixme}                  	% comments
\usepackage{ifthen}

% === setspace ===
\linespread{1.15}

% === tikz ===
\usetikzlibrary{
    fit,
    positioning,
    shapes,
    calc,
    pgfplots.groupplots,
    plotmarks,
    backgrounds,
    matrix,
    fadings,
    external,
    decorations.markings,
    patterns,
    arrows.meta,
    intersections}

% === xcolor ===
% Color set "strong" of TU Dortmund university
\definecolor{tuGreen}{cmyk}{0.57, 0, 1, 0}
\definecolor{tuOrange}{cmyk}{0, 0.55, 1, 0}
\definecolor{tuBlue}{cmyk}{1, 0, 0.36, 0}
\definecolor{tuViolette}{cmyk}{0.25, 1, 0, 0}
\definecolor{tuYellow}{cmyk}{0.05, 0.08, 1, 0}
\definecolor{tuBrown}{cmyk}{0.05, 0.69, 1, 0}

% === pgfplots ===
% Definition of cycle lists
\pgfplotscreateplotcyclelist{TUDefault}{%
    {plain plot set = {none}{tuGreen}},%
    {plain plot set = {none}{tuOrange}},%
    {plain plot set = {none}{tuBlue}},%
    {plain plot set = {none}{tuViolette}},%
    {plain plot set = {none}{tuYellow}},%
    {plain plot set = {none}{tuBrown}}%
}

\pgfplotsset{
    compat = newest,
    scale only axis,
    width = 0.85\textwidth,
    height = 0.25\textheight,
    every axis legend/.append style = {nodes={right}},
    xmajorgrids,
    ymajorgrids,
    legend style = {font = \tiny, at = {(0.5,1.025)}, anchor = south, legend columns = 6},
    % Custom plot style (https://tex.stackexchange.com/questions/193589/pgfplots-how-can-i-define-one-cycle-list-for-all-graphs-or-the-least-amount-p/193596)
    plain plot set/.style 2 args = {
        #2,
        mark = #1,
        every mark/.append style = {fill = #2}
    },
    % Axis labels according to DIN 461 (based on https://tex.stackexchange.com/questions/391332/get-the-auto-generated-tick-distance-of-a-pgfplot-din-461/391430#391430)
    % Argument 1: Axis description
    % Argument 2: Axis unit
    din xlabel/.style 2 args={
        xticklabel style={
            name=xlabel\ticknum,
            append after command=\pgfextra{\xdef\lastxticknum{\ticknum}}
        },
        after end axis/.append code={
            \pgfmathparse{int(\lastxticknum-1)}
            \path (xlabel\lastxticknum.base) -- (xlabel\pgfmathresult.base) node (xAxUnit) [midway, anchor=base] {#2};
            \node[anchor = east] (xAxDescr) at ($(xlabel\lastxticknum.base) + (-5mm,-3mm)$) {#1};
            \draw[->] (xAxDescr.east) -- ($(xlabel\lastxticknum.base) + (0,-3mm)$);
        }
    },
    din ylabel/.style 2 args={
        yticklabel style={
            name=ylabel\ticknum,
            append after command=\pgfextra{\xdef\lastyticknum{\ticknum}}
        },
        after end axis/.append code={
            \pgfmathparse{int(\lastyticknum-1)}
            \path (ylabel\lastyticknum.base) -- (ylabel\pgfmathresult.base) node (yAxUnit) [midway, anchor=base] {#2};
            \node[anchor = east, rotate = 90] (yAxDescr) at ($(ylabel\lastyticknum.west) + (-3mm,-5mm)$) {#1};
            \draw[->] (yAxDescr.east) -- ($(ylabel\lastyticknum.west) + (-3mm,0)$);
        }
    },
    din zlabel/.style 2 args={
        zticklabel style={
            name=zlabel\ticknum,
            append after command=\pgfextra{\xdef\lastzticknum{\ticknum}}
        },
        after end axis/.append code={
            \pgfmathparse{int(\lastzticknum-1)}
            \path (zlabel\lastzticknum.base) -- (zlabel\pgfmathresult.base) node (zAxUnit) [midway, anchor=base] {#2};
            \node[anchor = east, rotate = 90] (zAxDescr) at ($(zlabel\lastzticknum.west) + (-3mm,-5mm)$) {#1};
            \draw[->] (zAxDescr.east) -- ($(zlabel\lastzticknum.west) + (-3mm,0)$);
        }
    }
}

\usepgfplotslibrary{groupplots}
\pgfdeclarelayer{background}
\pgfdeclarelayer{foreground}
\pgfsetlayers{background,main,foreground}
 
% === SIunitx ===
\sisetup{
    load-configurations = abbreviations,            % load units with abbreviations
    range-phrase = \ldots,
    list-final-separator = { \translate{und} },
    list-pair-separator = { \translate{und} },
    output-decimal-marker = {,},
    group-minimum-digits = 3,
    group-separator = {.},
    exponent-product = \cdot,
    per-mode = symbol-or-fraction
}
\DeclareSIUnit{\Wp}{Wp}
\DeclareSIUnit{\kWp}{kWp}
\DeclareSIUnit{\kWh}{kWh}
\DeclareSIUnit{\GWh}{GWh}
\DeclareSIUnit{\TWh}{TWh}
\DeclareSIUnit{\pu}{p.u.}
\DeclareSIUnit{\VA}{VA}
\DeclareSIUnit{\VAr}{VAr}

% Command for printing the complex j
\newcommand{\cj}{\text{j}}

\begin{document}
	    \begin{tikzpicture}
	    % === Anschlüsse ===
        \node[circle, draw = black, inner sep = 0, minimum height = 1.5mm] (port_a_+) at (0,0){};
        \node[circle, draw = black, inner sep = 0, minimum height = 1.5mm] (port_a_-) at (0,-15mm){};
        \node[circle, draw = black, fill = black, inner sep = 0, minimum height = 0.75mm] (port_h_+) at (30mm,0){};
        \node[circle, draw = black, fill = black, inner sep = 0, minimum height = 0.75mm] (port_h_-) at (30mm,-15mm){};
        \node[circle, draw = black, inner sep = 0, minimum height = 1.5mm] (port_c_+) at (60mm,0){};
        \node[circle, draw = black, inner sep = 0, minimum height = 1.5mm] (port_c_-) at (60mm,-15mm){};
        \node[circle, draw = black, inner sep = 0, minimum height = 1.5mm] (port_b_+) at (75mm,15mm){};
        \node[circle, draw = black, inner sep = 0, minimum height = 1.5mm] (port_b_-) at (75mm,-15mm){};
        \coordinate (port_b2_+) at ($(port_h_+) + (0,15mm)$);
	
	    % === Bauteile ===        
        % Passive Elemente
	    \node[rectangle, draw = black, inner sep = 0, minimum height = 3.5mm, minimum width = 10mm, label={90:\texttt{rScA, xScA}}] (yka) at ($(port_a_+)!0.6!(port_h_+)$){};
	    \node[rectangle, draw = black, inner sep = 0, minimum width = 3.5mm, minimum height = 10mm, label={[label distance = 1.5mm, anchor = east]180:\texttt{gM, bM}}] (yh) at ($(port_h_+)!0.5!(port_h_-)$){};
	    \node[rectangle, draw = black, inner sep = 0, minimum height = 3.5mm, minimum width = 10mm, label={90:\texttt{rScB, xScB}}] (ykb) at ($(port_b2_+)!0.5!(port_b_+)$){};
	    \node[rectangle, draw = black, inner sep = 0, minimum height = 3.5mm, minimum width = 10mm, label={90:\texttt{rScC, xScC}}] (ykc) at ($(port_h_+)!0.4!(port_c_+)$){};
	    
	    % Verbindungen
	    \draw (port_a_+.east) -- (yka.west) (yka.east) -- (port_h_+.west);
	    \draw (port_h_+.south) -- (yh.north) (yh.south) -- (port_h_-.north);
	    \draw (port_h_+.east) -- (ykc.west) (ykc.east) -- (port_c_+.west);
	    \draw (port_h_+.east) -| (port_b2_+) |- (ykb.west) (ykb.east) -- (port_b_+.west);
	    \draw (port_a_-.east) -- (port_h_-) -- (port_c_-.west) (port_c_-.east) -- (port_b_-.west);
	    
	    % Erde
	    \draw (port_h_-) -- ++(0,-2.5mm) ++(-2mm,0) -- ++(4mm,0) ++(-0.5mm,-1mm) -- ++(-3mm,0) ++(0.5mm,-1mm) -- ++(2mm,0);
	    
	    % Spannungs- und Strompfeile
	    \node[anchor = east, inner sep = 2mm] at (port_a_+.west) {\texttt{nodeA}};
	    \node[anchor = west, inner sep = 2mm] at (port_b_+.east) {\texttt{nodeB}};
	    \node[anchor = west, inner sep = 2mm] at (port_c_+.east) {\texttt{nodeC}};
	    \end{tikzpicture}
\end{document}